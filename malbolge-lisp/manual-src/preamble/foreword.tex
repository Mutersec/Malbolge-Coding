
\chapter*{Foreword}
\par MalbolgeLISP is currently one of my most popular projects. It has recently gained traction on sites like Reddit\footnote{\url{https://www.reddit.com/r/lisp/comments/oxtpnn/kspalaiologosmalbolgelisp_a_lightweight_150mb/}}, HackerNews\footnote{\url{https://news.ycombinator.com/item?id=28048072}} GitHub\footnote{\url{https://github.com/kspalaiologos/malbolge-lisp}, with around 60'000 visits within the first few days of publishing and keeping the profile of 60 unique visit each day after a few weeks.}, \textit{surprisingly Turing-complete}\footnote{\url{https://www.gwern.net/Turing-complete}} and many others. During the fall of 2020, as well as the summer and fall of 2021, I spent my time optimising my toolkit, working on the Malbolge interpreter and MalbolgeLISP, troubleshooting performance problems, implementing new features, and testing them.

\par In 2021, two new versions of MalbolgeLISP were released - v1.1 and v1.2, the first one featuring performance optimisations and a few additional features, and the second one featuring many functional devices and a fast Malbolge interpreter. Across the releases, the codebase became harder to test (lack of testing hardware, limited amounts of test programs) and harder to work with (increasingly long compilation times, larger code size, more complicated compilation pipeline, since I introduced new features to my toolchain while developing MalbolgeLISP, etc...). In particular, I'd like to thank Github user Matt8898\footnote{\url{https://github.com/Matt8898/}} for providing me test cases and feedback.

\par During the course of this project, I've been drawing inspiration from APL (Dyalog APL\footnote{\url{https://www.dyalog.com/}} in particular) and Haskell\footnote{\url{https://www.haskell.org/}}. Their influence is evident judging by the function set and syntax extensions to traditional Lisp, while the APL influence is additionally emphasised by providing equivalent expressions to MalbolgeLISP built-in operations or example expressions, as well as explaining the core concepts of MalbolgeLISP using APL as a tool of thought.
